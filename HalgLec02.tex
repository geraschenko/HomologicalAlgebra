\sektion{2}{???}

\subsektion{Basics of chain complexes}
I'll write down a definition, you tell me what it means. Fix some associative ring $R$, and we'll look at the category $R\mod$ (left or right). You can play this game for abelian categories, but let's not do that now.

A sequence
\[
 M_1\xrightarrow f M_2 \xrightarrow g M_3
\]
is \emph{exact} (at $M_2$) if $\im f = \ker g$. Note that this is more than saying $g\circ f =0$ (which just says $\im f\subseteq \ker g$). If $M_3=0$, then this is exact if $f$ is onto. If $M_1$ is zero, the sequence is exact if and only if $g$ is injective.

A \emph{long exact sequence} (LES) is a sequence
\[
 \cdots \to M_1\to M_2\to M_3\to M_4\to \cdots
\]
which is exact at each place. Note that a LES is a chain complex (actually, since the indices go up, we call it a \emph{cochain complex}\footnote{To get from a chain complex to a cochain complex, you could apply any contravariant additive functor, or you could just re-index (define $M_i=C_{N-i}$ for any $N$). Another way is to define $M_i=C_i^*=\hom_R(C_i,R_i)$, but there is a problem. If $C_i$ are left $R$-modules, then $C_i^*$ are right $R$-modules. Let $R\mod$ be the category of left modules, and let $\rmod R$ be the category of right modules. So we have a functor $\hom_R(-,R)\colon \rmod R^\circ\to R\mod$ (the circle means opposite category). Poincar\'e duality says that for a closed oriented manifold of dimension $N$, the two cochain complexes $C_i^*$ and $M_i$ are homotopy equivalent. So $H^i(M)\cong H_{n-i}(M)=H_i(C_{n-*}(M))$.}). In fact, a LES is exactly a (co)chain complex $M_*$ with $H^i(M_*)=0$ for all $i$. This is also called an \emph{acyclic} (co)chain complex.

Another thing that comes up a lot is a \emph{short exact sequence} (SES), which is an exact sequence of the form
\[
 0\to M_1\xrightarrow f M_2 \xrightarrow g M_3\to 0
\]
The sequence is exact at all places, so $f$ is injective, $g$ is surjective, and $\im f = \ker g$.

Given any chain complex $C_*=(\cdots C_2\to C_1\xrightarrow{d_1} C_0\to C_{-1}\to \cdots)$, we have $Z_i=\ker d_i\subseteq C_i$, the \emph{$i$-cycles}\footnote{It's a $Z$ because in another language, ``cycle'' is ``zykel''.}, and $B_i=\im d_{i+1}\subseteq C_i$, the \emph{$i$-boundaries}. Because it is a complex, $B_i\subseteq Z_i$. The homology is $H_i(C_*)=Z_i/B_i$.

Let me draw some pictures so remind you where this terminology comes from.
\begin{example}[Poincar\'e]
 Let $K$ be a simplicial complex, i.e.~$K$ is a union of simplices along faces in some $\RR^N$ (I don't want to make this precise because I don't think this is a nice notion\footnote{Later, we'll make this stuff more precise with simplicial sets.}). We want to define the homology of this simplicial complex. We define $C_n(K)$ to be the free abelian group generated by the $n$-simplices. Choose some ordering on the vertices, so $C_0$ has ordered bases. We orient the edges (going from the smaller number to the bigger number). Similarly, we orient the 2-simplices and so on. Now what are the boundary maps?
 \[
  \anton{picture}
 \]
 \[
  \cdots \to C_2(K)\to C_1(K)\to C_0(K)
 \]
 First, let's decide what the boundary of an edge is. Let's say $d(ij)=j-i=1\cdot j + (-1)\cdot i\in C_0(K)$ (you have to have the signs on the boundary pieces to get $d^2=0$). For 2-simplices, we read off the edges on the boundary with sign determined by whether you go along the orientation of the edge or against it. So $d(ijk)=jk-ik+ij$.
 
 Now you can kind of see why things in $B_i$ are called boundaries (because you take some linear combination of boundary simplices). To see why $Z_i$ are called cycles, consider $Z_1$. To get the boundary to be zero, every time you have an edge coming into a vertex, you have to have an edge going out of it.
\end{example}

[[break]]

\subsektion{Quick review of CW-complexes}

A \emph{CW-complex} $X$ is a Hausdorff space with a decomposition (as a set) $X=\bigsqcup_n \bigsqcup_{i\in I_n} e_i^n$, where $e_i^n\cong \RR^n$ are $n$-cells (simplices, if you remove the boundary, are $n$-cells) such that there exist continuous maps $\phi_i^n\colon D^n\to X$ such that $\phi_i^n\colon \inn D^n\cong e_i^n$ such that
\begin{itemize}
 \item[(C)] (closure finite) $\phi_i^n(S^{n-1})$ is contained in a finite number of $e_i^k$ for $k<n$, and
 \item[(W)] (weak topology) $A\subseteq X$ is closed (or open) if and only if $(\phi_i^n)^{-1}(A)\subseteq D^n$ is closed (or open).
\end{itemize}
The intuitive thing to remember is that $X$ is a disjoint union of open cells whose boundaries meet only a finite number of lower-dimensional cells.
\begin{lemma}
 If $X$ and $Y$ are CW-complexes, then $(X\times Y)_{cg}$ (compactly generated) is a CW-complex with cells $e_X^p\times e_Y^q$ for $p+q=n$.
\end{lemma}


Now we can define the \emph{cellular chain complex} $C_*(X)$. Take $C_n(X)$ to be the free abelian group on the $n$-cells, with $d(e^n)$ a linear combination (with correct signs) of the $(n-1)$-cells that appear in $\phi_i^n(S^{n-1})$. This is not precise, but if you took algebraic topology, you know how to make this precise. Q: how do you remember the signs (in simplicial sets, we just ordered the vertices and that took care of it)? PT: you orient each cell however you like and you just deal with those orientations.

Assume that a group $G$ acts (on the right) on our CW-complex $X$, permuting the cells freely (i.e.~$g(e^n_i)=e^n_j$, with $i\neq j$ if $g$ not the identity). 
\begin{example}
 Take $X=\RR$ with a vertex at each integer and 1-cells in between (btw, you can't think of it as a single 1-cell because there is no map $D^1\to \RR$; in fact, you can prove that if you have finitely many cells, the CW-complex is compact) and take $G=\ZZ$, acting by translation. 
\end{example}
\begin{example}
 $X=\RR^n$ with $G=\ZZ^n$ by taking the product of the previous example with itself $n$ times.
\end{example}
\begin{lemma}
 $X/G$ is again a CW-complex with an $n$-cell for each $G$-orbit of $n$-cells in $X$.
\end{lemma}
\begin{theorem}
 If $X$ is contractible with such a $G$-action, then $H_*(X/G)\cong H_*(G)$ (defined by $\ZZ G$ projective resolutions of $\ZZ$).
\end{theorem}
The upshot is that you can use topological homology to compute group homology.









