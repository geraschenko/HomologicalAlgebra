\sektion{28}{???}

\begin{lemma}
 Let $A$ be an object in an abelian category $\A$, and let $\Sigma^n A\in \chain(\A)$ be $A$ concentrated in degree $n$.
 \begin{itemize}
  \item[(a)] $[\Sigma^n A,C_*]\cong H_n(\A(A,C_*))$, which is isomorphic to $\A(A,H_n C_*)$ if $A$ is projective or $C_*$ is a complex of injectives.
  \item[(b)] $[C_*,\Sigma^n A]\cong H_n(\A(C_*,A))=:H^n(C_*;A)$, which is isomorphic to $\A(H_n C_*,A)$ if $A$ is injective or $C_*$ is a complex of projectives.
 \end{itemize}
\end{lemma}
You can think of $\Sigma^n A$ as the ``$A$-$n$-sphere'' by part (a) (interpreting $H_n$ as homotopy groups), or you can think of it as a $K(A,n)$ (which should represent cohomology) by part (b).
\begin{proof}
 (a) A chain map from $\Sigma^nA$ to $C_*$ is a map $f\colon A\to C_n$ such that $d\circ f=0$:
 \[\xymatrix{
   0\ar[d]\ar[r] & A\ar[r] \ar[d]^f & 0 \ar[d]\\
   C_{n+1}\ar[r] & C_n \ar[r]^d & C_{n-1}
 }\]
 Thus, a chain map is exactly an element of $Z_n(\A(A,C_*))$. A nulhomotopy of $f$ is a map $h\colon A\to C_{n+1}$ such that $f=dh$. So nulhomotopies are exactly elements of $B_n(\A(A,C_*))$. Thus, we get part (a).

 Part (b) is just as easy.
\end{proof}
\begin{corollary}
 $\Der^-(\A)(\Sigma^{-n}A,B)\cong \Der^-(\A)(A,\Sigma^nB)\cong \ext^n_\A(A,B)$.
\end{corollary}
\begin{proof}
 $\Der^-(\A)(\Sigma^{-n}A,B)=h\chain^-(\A)(\Sigma^{-n},I_*)=:\footnote{In the derived category, this is really supposed to be $h\chain(\A)(I_A,I_B)$, but we only need to resolve $B$ because $I_B$ is local, so $h\chain(\A)(I_A,I_B)\cong h\chain(\A)(A,I_B)$.}[\Sigma^{-n}A,I_*]$, where $I_*$ is an injective resolution (i.e~a localization) of $B$. By the Lemma, this is $H_{-n}(\A(A,I_*))=:\ext^n_\A(A,B)$. Note that the injective resolution is number in a slightly funny way to conform to our convension of the differentials always decrease degree: $0\to B\to I_0\to I_{-1}\to \cdots$.
\end{proof}
\begin{remark}
 Recall that we had the Yoneda product on $\ext$ given by splicing (if you think of $\ext$ as parameterizing extensions). This just corresponds to composition in the derived category: $\Sigma^{-n-m}A\xrightarrow{\Sigma^{-m} f} \Sigma^{-m}B\xrightarrow{g} C$.
 
 Recall that $\ext^n(A,B)$ is generated by $\ext^1$ (because any long exact sequence in an abelian category can be ``unspliced'' into a bunch of short exact sequences.). This implies that $\Der^-(\A)^n(A,B):=\Der^-(\A)(\Sigma^{-n}A,B)$ is generated by morphisms of degree 1.
 
 If you have an arbitrary triangulated category, a necessary criterion for it to be the derived category of some abelian category is for the graded $\hom$ to be generated by things of degree 1. Sometimes, this condition is sufficent.
\end{remark}
After the break, we'll switch from local objects to colocal objects. This will show us why CW complexes are so special. But first, let's review group homology.

I keep changing HW13. I also changed HW12 to fix the Hurewicz problem. It is actually now more general; it shows that homotopy groups of finite simply connected CW complexes are finitely generated.

\subsektion{Review of group (co)homology}
(a) We always assumed $G$ is discrete; later, I'll talk about what happens if $G$ has a topology. We defined $H_n(G)=H_n(K(G,1))$, but this was not so good because it wasn't functorial, so we used the bar complex. Then we actually find a functorial model of $K(G,1)$, namely $BG$. Recall that $BG:=|N_\udot \C_G|$ was defined as the geometric realization of the nerve of the category associated to $G$.
\begin{remark}
 You can cicumvent topological spaces; everything is totally combinatorial.
 \begin{align*}
  H_n(G)&\cong H_n(|N_\udot \C_G|)\\
  &\cong H_n(Alt_* Free(N_\udot \C_G))\\
  &\cong H_n(N_* Free(N_\udot \C_G))\qedhere
 \end{align*}
\end{remark}
Recall that $N_n \C$ are $n$-tuples of composible morphisms, with degeneracy maps given by interting identities and face maps given by composing two consecutive maps. Thus, $N_n \C_G\cong G^{n+1}$.

Now suppose $A$ is a trivial $G$-module, then how would we compute $H^n(G;A)$? We would look at the complex $\ab(\ZZ[G^n],A)\cong \set(G^n,A)$. This gives us the usual cocyle/coboundary definition of group cohomology.

If $A$ is a non-trivial $G$-module, there is a totally analogous analysis. You have to know that the universal cover of $BG$ is $EG=|N_\udot \D_G|$, where $\D_G$ is the transport category on $G$ (more generally, if $G$ acts on $X$, we get a category whose objects are $X$ and whose morphisms are $G\times X$). Recall from Hochschild cohomlogy that we had a simplicial $G$-bimodule, and we quotiented on one side by the $G$ action to get the Bar complex. We need to compute the homology group $H^n(G;A)\cong H_n(G\mod(\ZZ[G^{*+1}],A)\cong H_n(\set(G^*,A))$. The last step works because $\ZZ[G^{*+1}]$ is a free $\ZZ [G]$-module. Again, you get $n$-cocycles in $A$ modulo $n$-coboundaries in $A$. The only difference is that the alternating sums of things depend on the action, where they didn't before.
\begin{remark}
 This approach genralizes to groupoid homology, category homology (just replace $\C_G$ by your favorite small category), $n$-groupoids (and $n$-categories), and topological groups. If you have a topology on the objects and morphisms of a category, then the nerve is a simplicial topological space (not just a simplicial set). But if you think about geometric realization, $|X_\udot| = \bigsqcup \Delta^k\times X_k/\sim$, you may as well use the topology on $X_k$ in the definition.
 
 If $G$ is any topological group, you get $BG$ and $EG$ as before, and it still turns out that $EG\cong *$ and $EG\to BG$ is a $G$-principal bundle. A lemma shows that this is the universal $G$-bundle (i.e.~isomorphism classes of $G$-principal bundles on a CW complex $X$ are parameterized by $[X,BG]$). If you apply this to the discrete case, you see that maps to a $K(G,1)$ (i.e.~$BG$) parameterize covering spaces with fiber $G$.
\end{remark}
\begin{warning}
 If $G$ is not discrete, $BG$ is \emph{not} a $K(G,1)$, because $\pi_n(BG)\cong \pi_{n-1}(G)$ (because $EG$ is contractible). Thus, the homology of $BG$ doesn't agree with the homology of $G$.
\end{warning}
If $G$ is a topological group and if $A$ is a $G$-module (with a topology), then you can define continuous cohomology. Similarly, you can define smooth cohomology or Borel cohomology (if you have measures).

[[break]]

\subsektion{Survey on colocalizations}
Recall that we had a subcategory of equivalences $\E\subseteq \C$. We had a nice construction if we have all localizations. There is an obvious concept of \emph{colocal} objects: $X$ is colocal if $\C(X,A)\xrightarrow\cong \C(X,B)$ for any equivalence $A\xrightarrow\sim B$. We say that $\C$ has all cololcaizations if for every object $X$, there is an equivalence $C_X\to X$ where $C_X$ is colocal. If you have all colocalizations, you get the same story. The principle is that if $\C$ has all $\E$-colocalizations, then $\E^{-1}\C$ exists and the composition $CoLoc_\E \C\subseteq \C\to \E^{-1}\C$ is an equivalence of categories. You define the morphisms $\E^{-1}\C(X,Y)=\C(C_X,C_Y)$. Furthermore, for any functor $F\colon F\to D$, there exists a left derived functor $LF$:
\[\xymatrix{
 \C\ar[d]_F^{}="b" \ar[r] & \E^{-1}\C\ar[dl]^{LF}_{}="a"\\
 \D
 \ar@{=>} "a";"b"
}\]

\begin{theorem}
 If $\C=h\top$ and $\E$ consists of weak equivalences, then CW complexes are colocal and all colocalizations exist!
\end{theorem}
\begin{proof}
 By definition of weak equivalence, spheres a colocal. $X^{(n)}$ is the mapping cone on the attaching map $\bigsqcup S^{n-1}\to X^{(n-1)}$. Then use the long exact sequence for mapping cones, then by induction on the $n$-skeleton, you get that it is local.
 
 Weak equivalence exist because $|\Delta_\udot X|\to X$ is a weak equivalence from a CW complex.
\end{proof}
\begin{corollary}
 The derived category $\Der(h\top)\cong \Der(\top)$ of topological spaces is equivalent to $h\cw$.
\end{corollary}