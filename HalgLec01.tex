\sektion{1}{???}

This will be a \emph{basic} course in homological algebra, which is used in various areas. We won't focus on topology and we won't get too fancy. There will be homework every week, to be done in groups of two or three (no groups of one). I will give you very precise definitions and some intuition from geometry, but you have to do calculations for yourself. The most interesting stuff will be on the homework. Chris will be your GSI. We will grade only one of the problems, so you only have to write up one. There will be a meeting once a week with Chris where you'll have to present solutions. The purpose of this is to learn to present mathematics. The grade will be based on the written and presented homework. Each of us will have one office hour. Mine will be Tuesday 2-3.

A little bit of history. The origins of homological algebra. There are two main routes. 

One due to Poincar\'e, which might be called ``combinatorial topology'' today. The goal is to understand qualitative features of spaces. At the time, they didn't have nice definitions for topological spaces. Poincar\'e thought of them as built out of simplices. Some of the features are
\begin{enumerate}
 \item[0.] number of connected components,
 \item number of holes in a plane,
 \item number of holes in 3-space.
\end{enumerate}

Today, we'd measure these with Betti numbers; the number of components is $b_0$, the number of holes in a plane is $b_1$ (the holes are ``caught'' by loops), and the number of holes in 3-space (``caught'' by spheres) is $b_3$. The $n$-th Betti number is $b_n(X)=\rk H_n(C_* X)$.

Today, we'd say that Poincar\'e started with a topological space and constructed a chain complex, whose homology we can look at: $\top\to \chain\xrightarrow{H_i}\ab$ (the first arrow is sometimes not a functor)
\begin{definition}
 A \emph{chain complex} is a sequence of abelian groups $C_*=(\cdots C_{1}\xrightarrow{\partial_{1}} C_0\xrightarrow{\partial_0} C_{-1}\xrightarrow{\partial_{-1}}\cdots)$ such that $\partial_{i+1}\circ \partial_i=0$ for all $i$. The \emph{$n$-th homology} is $H_n(C_*)=\ker(\partial_n)/\im(\partial_{n+1})$.
\end{definition}
The tricky part is to get the chain complex. That's the part I'd call the art form. The main focus of the class, on the other hand, is to start with chain complexes. But sometimes we'll get some intuition from topology for why we do certian things with chain complexes. What Poincar\'e found is that even though the chain complex of the space is not an invariant, the homology is.

The other approach is due to Hilbert, and it lead to commutative algebra. He studied ideals $I\subseteq k[x_1,\dots, x_n]=R$. Let's assume $I$ is generated by finitely many homogeneous polynomials. That is, we have a surjection of $R$-modules $R^m\twoheadrightarrow I$. Then Hilbert defined the \emph{syzygy} $Z_0(I):=\ker(R^m\twoheadrightarrow I)$. This is measuring relations between the generators of $I$. Then Hilbert repeated this construction (you have to prove $Z_0$ is finitely generated ... it is). Let $F_0=R^m$.
\[\xymatrix{
 & 0\ar[r] & F_n \ar[dr]_@{~} &\cdots & F_2\ar@{->>}[dr]\ar[r]^{\partial_2} & F_1\ar@{->>}[dr] \ar[r]^{\partial_1}& F_0\ar@{->>}[r] & R/I\\
 &&& Z_{n-1} & Z_2\ar@{^(->}[u] & Z_1\ar@{^(->}[u] & Z_0\rlap{$\;=I$}\ar@{^(->}[u]
}\]
Hilbert's theorem is that this terminates: $Z_{n-1}$ is free! You all know this for $n=1$: $k[x]$ is a PID. Maybe I should have talked about $R/I$ instead of $I$ to get the indices shifted so that they are right. The generalization of being a PID is that any module has a free resolution of length $n$. We say that the ring has \emph{homological dimension $n$} in this case.

$F_*=(F_n\to \cdots \to F_0)$ is a chain complex (check that the compositions are zero!). The great thing about homological algebra is that it fits in the plane. What are the homology groups of $F_*$? I claim that
\[
 H_i(F_*)=\begin{cases}
  R/I & i=0\\
  0 & i\neq 0
 \end{cases}
\]
This is what it means to be a free resolution of $R/I$; a complex of free modules with this property. Since this is an iteration, it is enough to check that $H_1=0$. Well, $\ker \partial_1=Z_1=\im \partial_2$.

Here, we've used $R$-modules, not just abelian groups. What kind of other categories have chain complexes? An important one is sheaves of abeilan groups. You can make chain complexes whenever you can make sense of the chain condition $\partial_{i-1}\circ \partial_i=0$. For example, you can do this in additive categories. But to define homology groups, you need to have kernels and images. So the categories that people study chain complexes in are \emph{abelian categories}. We'll generalize this in the third part of this class, when we talk about derived categories (or model categories).

This ends the historic introduction to homological algebra.

[[break]]

There are some pretty good references. The references and homework will be posted on the website. Now let's start with Part 1: Group homology.

\subsektion{Group homology}

These days, you can form homology of anything (spaces, algebras, etc.). We'll focus on one thing: groups. We'll define homology functors $H_n\colon \gp\to \ab$ (there won't be any negative homology groups).

$H_0(G)$ is always $\ZZ$. $H_1(G)=G^{ab}=G/[G,G]$. The higher cohomology groups are not that easy. You can get them in two different ways, following Poincar\'e (there is a way to make a space from a group) or following Hilbert (using the module $\ZZ$ over $\ZZ G$). The module structure on $\ZZ$ is given by the augmentation map $\ZZ G\to \ZZ$, given by $\sum a_n g_n\mapsto \sum a_n$. We go to chain complexes by applying the syzygy construction to get a free resolution $F_*$ of $\ZZ$ (which will not terminate in general). Q: but then there aren't any interesting homology groups. PT: you're right, I forgot to apply a functor. You take coinvariants of the free resolution. If $M$ is a $G$-module (i.e.~there is a group homomorphism $G\to \Aut(M)$. This is the same as a $\ZZ G$-module). The \emph{invariants} are $M^G=\{m\in M|g\cdot m=m \forall g\in G\}$, the largest submodule on which $G$ acts trivially. The coinvariants are $M_G=M/\<g\cdot m-m\>_{\text{submod}}$, the largest quotient on which $G$ acts trivally. Coinvariants is not an exact functor, so we get some interesting cohomology. Note that $(\ZZ G)_G\cong \ZZ$ (this is actually the augmentation map). In this free resolution, all the terms are $\ZZ G^n$, and coinvariants behaves well with direct sums, so $(\ZZ G^n)_G=\ZZ^n$. You can make $F_1$ finitely generated if and only if $G$ is finitely generated and you can make $F_2$ finitely generated if and only if $G$ is finitely presented. If all the $F_i$ are finitely generated, then they are $\ZZ G^n$ for some $n$.

What is group homology good for? What is the higher group homology measuring? It turns out you can define cohomology as well (modules are like sheaves). $H^2(G;M)$ is in bijective correspondence with extensions $1\to M\to \tilde G\to G\to 1$ (modulo the obvious equivalence relation). You know that finite simple groups are classified. Does this mean that we understand all finite groups? Well, if a group is not simple, then there is a proper normal subgroup $N\subseteq Q$. Then $N$ and $Q/N$ are of smaller order, so we might understand them by induction, but we also have to understand the extension $1\to N\to Q\to Q/N\to 1$. These extensions will be parameterized by $H^2(Q/N;Z(N))$. $Q/N$ acts on the center of $N$ by pulling back and conjugating (?). The example I prepared, but don't have time to do, is a classification of groups of order $8$ (I would have told you the relevant cohomology groups).

First we'll define group cohomology and understand them up to $H^3$. Then we'll do an application: free group actions on spheres (we'll prove that a group acting freely on a sphere of dimension $n$ must have periodic cohomology with period $n+1$).

Part 2 will be about computational tools for $H_*(G;M)$:
\begin{itemize}
 \item Mayer-Vietoris sequences calculate cohomologies of things like $G_1\ast_H G_2$.
 \item K\"unneth theorem for $G_1\times G_2$.
 \item long exact sequences from short exact sequences of coefficient modules (just like in sheaf cohomology).
 \item Leray-Serre spectral sequence, which computes the homology of an extension in terms of the smaller pieces. This is actually the main tool for computing these homologies.
\end{itemize}
