\sektion{13}{Hochschild (co)homology}

Recall Hochschild (co)homology. Given a $k$-algebra $R$ ($k$ could be $\ZZ$ if $R$ is just a ring), we get an (augmented) semi-simplicial $(R,R)$-bimodule
\[\xymatrix@R-1.5pc{
  & 1 & 0\\
 \cdots \ar[r]|-{d_1}\ar@<1ex>[r]^-{d_0}\ar@<-1ex>[r]_-{d_2} & R\otimes R\otimes R\ar@<.5ex>[r]^-{d_0}\ar@<-.5ex>[r]_-{d_1} & R\otimes_k R\ar[r]^-\mu & R
}\]
where $d_i(a_0\otimes \cdots \otimes a_n)= a_0\cdots \otimes a_ia_{i+1}\otimes \cdots \otimes a_n$.

Recall that for any category $\C$, we have $\presh(\C)=\fun(\C^\circ,\set)$, and we defined semi-simplicial sets as objects of $\presh(\C_\Delta)$. A semi-simplicial object in $\A$ is an object in $ss\A=\fun(\C_\Delta^\circ, \A)$. The objects of $\C_\Delta$ are $[n]=\{0,\dots, n\}$ with $n\ge 0$, and the morphisms are strictly order preserving set maps. The morphisms are generated by the face maps $\partial_i\colon [n-1]\to [n]$ (the ``skip $i$'' map) for $0\le i\le n$. Any morphism $f\colon [m]\to [n]$ has a unique decomposition $f=\partial_{i_1}\circ\cdots\circ \partial_{i_k}$ with $i_1>\cdots >i_k$. The relations are $\partial_j\circ \partial_i=\partial_i\circ\partial_{j-1}$ for $i<j$.

So a semi-simplicial object in $\A$ is a sequence of objects $X_n$ (the images of the $[n]$) and morphisms $d_i\colon X_n\to X_{n-1}$ (the images of the $\partial_i$, which go the other way), satisfying the relations $d_i\circ d_j=d_{j-1}\circ d_i$ when $i<j$.

\begin{lemma}[``main lemma of homological algebra'']
 If $X_\udot$ is a semi-simplicial object in an abelian category $\A$ (an object in $ss\A$), then $(X_*,d)$ is a chain complex, where $X_n=X_n$ and $d=\sum_{i=0}^n (-1)^i d_i$.
\end{lemma}
\anton{Dold-Kahn correspondence is some equivalence of categories to this effect} In particular, we get a chain complex of $(R,R)$-bimodules $T_*(R)$ (including the $-1$ piece). We showed that as a chain complex of left (or right) $R$-modules, $T_*(R)$ is contractible.

Let $M$ be a left $R$-module, then $T_*(R)\otimes_R M$ is an acyclic chain complex of left $R$-modules (it is contractible as a complex of abelian groups). That is, we have a resolution of $M$ by things of the form $R^{\otimes n+2}\otimes_R M\cong R^{\otimes n+1}\otimes_k M$.
\begin{lemma}
 If $R$ and $M$ are flat modules over $k$, then $T_*(R)\otimes_R M$ is an $R$-flat resolution of $M$ as a left $R$-module.
\end{lemma}
This is true just because for a right $R$-module $N$, $N\otimes_R R\otimes_k B\cong N\otimes_k B$, so if $B$ is flat as a $k$-module, $R\otimes_k B$ is flat as an $R$-module. You also have to use that the tensor product of flat modules is flat (because tensor product is associative).

So we may use the resolution $T_*(R)\otimes_R M$ to compute $\tor_n^R(N,M)$ for a right $R$-module $N$. More precisely,
\[
 \tor_n^R(N,M) \cong H_n(N\otimes_R T_*(R)\otimes_R M)= HH_n(R; {}_RM\otimes_k N_R)
\]
(here we are taking $*\ge 0$, not the augmented guy).

If $B$ is an $(R,R)$-bimodule, then recall that $HH_n(R;M)=H_n(T_*(R)\otimes_{R,R} B)$. If $A$ and $B$ are bimodules, then $A\otimes_{R,R}B = A\otimes_R B/(ra\otimes b-a\otimes br)$ (this is just an abelian group).

So we've proven the following.
\begin{proposition}
 If $R$ is $k$-flat and either $M$ or $N$ is $k$-flat, then  $\tor_n^R(N,M) \cong HH_n(R; {}_RM\otimes_k N_R)$.
\end{proposition}

\begin{theorem}
 $HH_n(\ZZ G; M_\e)\cong H_n(G;M)$ and $HH^n(\ZZ G;M_\e)\cong H^n(G;M)$.
\end{theorem}
\begin{proof}
 $H_n(G;M):=\tor_n^{\ZZ G}(\ZZ,M)$. Since $\ZZ$ and $\ZZ G$ are $\ZZ$-flat (torsion-free; in fact, they are free), we can apply the Proposition to tell us that $H_n(G;M)\cong HH_n(\ZZ G;M\otimes_\ZZ \ZZ)$, but $M_\e= M\otimes_\ZZ \ZZ$ by definition.
 
 \anton{The other assertion please} 
\end{proof}

[[break]]

\begin{remark}
 $HH_n(R;R)$ is given by the homology of $T_*(R)\otimes_{R,R}R$. Note that $R^{\otimes_k n+2}\otimes_{R,R} R\cong R^{\otimes_k n+1}$. This is sometimes just called the ``Hochschild homology of $R$, $HH_n(R)$''. You can think of it as $HH(R):=R\lotimes_{R,R} R$. The reason that the Hochschild homology of $R$ is the value of the circle in many 2-dimensional QFTs is that $\otimes_{R,R}$ ``loops together two bimodules''.
\end{remark}

\subsektion{Homotopy theory in homological algebra}

Remember from the homework that we have a pair of adjoint functors $\xymatrix{ss\set \ar@<.5ex>[r]^-{|\cdot |} & \top \ar@<.5ex>[l]^-{\Delta_\udot}}$, where $\Delta_n(X)=\{f\colon \Delta^n\to X\}$. 

For a semi-simplicial set $X_\udot$, you should think of elements of the set $X_n$ as the labels of $n$-simplices. Then the map $d_i\colon X_n\to X_{n-1}$ takes a label $x$ to the label of the $i$-th face of the simplex labelled by $x$. The relations come from the fact that there are two ways to get to a codimension 2 face by going through a codimension 1 face first.

(1) The homotopy groups $\pi_n(X)$ are actually the homotopy groups $\pi_n(\Delta_\udot(X))$ for some notion of $\pi_n$ of certain semi-simplicial sets. I'll explain this next time.

(2) $H_n^{\text{sing}}(X)=H_n(S_*(X))$, but $S_*(X)$ is the chain complex corresponding to $Free_\ab(\Delta_\udot(X))$. We'll see that in general, this is $\pi_n(Free \Delta_\udot(X))$.

So there is some notion of $\pi_n$ which in one instance gives $\pi_n(X)$ and in some other instance gives $H_n(X)$. The cool thing is that the Hurewicz map is the map on $\pi_n$ induced by $\Delta_\udot(X)\to Free(\Delta_\udot(X))$.