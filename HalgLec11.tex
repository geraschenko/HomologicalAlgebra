\sektion{11}{Extensions (of modules, rings, groups)}

Extensions are usually done with explicit cocylces in the Bar resolution. This doesn't make for a good lecture, so we'll do that in the homework.

\begin{theorem}
 Let $R$ be a ring, and let $M,N\in R\mod$. Then there is a natural bijection\footnote{In fact, it is an isomorphism of abelian groups. The zero element is the product. You can try figuring out how to ``add'' two extensions to get another extension.} $\ext^1_R(M,N)\cong \{$extensions in $R\mod$ of $M$ by $N\}/\sim$. An extension is a short exact sequence
 \[
  0\to N\to E\to M\to 0
 \]
 and two extensions $E$ and $E'$ are equivalent if there is an isomorphism which induces the identity on the subobject $N$ and the quotient $M$:
 \[\xymatrix@R-1pc{
  0\ar[r] & N\ar@{=}[d] \ar[r] & E\ar[d]^\wr \ar[r] & M\ar[r] \ar@{=}[d] & 0\\
  0\ar[r] & N\ar[r] & E'\ar[r] & M\ar[r] & 0
 }\]
\end{theorem}
\begin{example}
 $\ext^1_\ZZ(\ZZ/p,\ZZ)\cong \ZZ/p$. In this case, the exentions are given by the trivial extension $\ZZ\times \ZZ/p$ and the extentions $0\to \ZZ\xrightarrow p \ZZ \xrightarrow{1\mapsto k} \ZZ/p\to 0$, where $k\in (\ZZ/p)^\times$.
 
 Note that if we have the map $\ZZ/p\xrightarrow k\ZZ/p$, you can ``pull the extension back'' along this map. We can get all the extensions from the obvious extension $0\to \ZZ\xrightarrow p \ZZ\to \ZZ/p\to 0$ by pulling back along such maps (including $k=0$).
\end{example}
More generally, if we have a map $N\to N'$, we can push out the extension, and if we have a map $M'\to M$, we can pull it back.
\[\xymatrix{
 0\ar[r] & N\ar[d] \ar[r] & E\ar[d] \ar[r] & M\ar[r] \ar@{=}[d] & 0\\
 0\ar[r] & N'\ar[r] & P\ar[r]\ar@{}[ul]|(.25)\po & M\ar[r] & 0
}\]
Since we have the map $E\to M$ and $N'\xrightarrow 0 M$, we get a map $P\to M$. A diagram chase shows that you get a short exact sequence.
\[\xymatrix{
 0\ar[r] & N\ar@{=}[d] \ar[r] & Q\ar[d] \ar[r] \ar@{}[dr]|(.25)\pb & M'\ar[r] \ar[d] & 0\\
 0\ar[r] & N\ar[r] & E\ar[r] & M\ar[r] & 0
}\]
\begin{example}
 Let $G$ be a group. Then $\ext^1_{\ZZ G}(\ZZ,\ZZ)=: H^1(G;\ZZ)$. By the universal coefficient theorem \anton{see lecture 25 or so}, this is $\hom_{\ZZ\mod}(H_1(G),\ZZ)\cong \hom_\gp(G,\ZZ)$. Starting with such a homomorphism $\phi$, we can try to try to get an extension of $\ZZ$ by $\ZZ$ as a push out along $H_1 G\xrightarrow \phi \ZZ$.
 \[\xymatrix{
  0\ar[r] & H_1 G\ar[d]_\phi \ar[r] & ? \ar[r] \ar[d] & \ZZ\ar[r]\ar@{=}[d] & 0\\
  0\ar[r] & \ZZ \ar[r] & E\ar@{}[ul]|(.25)\po \ar[r] & \ZZ\ar[r] & 0
 }\]
 It turns out that there is a way to fill in the question mark so that any $E$ is obtained in as a pushout in this way.
 \[
  0\to I_G/I_G^2 \to \ZZ G/I_G^2 \xrightarrow \e \ZZ\to 0
 \]
 It turns out that $I_G/I_G^2\cong H_1(G)\cong G^{ab}$. This does the trick. \anton{how to see this?}
\end{example}
\begin{theorem}\label{11T:Ext^n}
 $\ext^n_R(M,N)$ is naturally in bijection with length $n$ extensions of $M$ by $N$ up to equivalence. A length $n$ extension is an exact sequence $E_*$
 \[
  0\to N\to E_n\to E_{n-1}\to \cdots \to E_1\to M\to 0.
 \]
 Two extensions $E_*$ and $E'_*$ are equivalent when there is a chain map (need not be an isomorphism) $E_*\to E'_*$ restricting to equality on $M$ and $N$. The whole equivalence relation is generated by these.
\end{theorem}
Again, you can push out and pull back extensions, so the naturality makes sense. When you push out along $N\to N'$, you can either push out all the $E_i$, or just $E_n$ (there is a map between the two, so they are equivalent). Similarly, when pulling back along $M'\to M$, you can either pull back all the $E_i$, or just $E_1$.

Let $K_i$ be the kernel of $E_i\to E_{i-1}$ (so $N=K_n$ and $M=K_0$). These break up the long exact sequence into a bunch of short exact sequences $0\to K_i\to E_i\to K_{i-1}\to 0$ as usual (with $1\le i\le n$).
\begin{proof}[Proof of Theorem \ref{11T:Ext^n}]
 Recall dimension shift for Ext. We can think of $E_*$ as giving a map $\ext^k_R(A,M)\to \ext^{k+n}_R(A,N)$. To do this, use the connecting maps $\ext^{k+i}(A,K_i)\to \ext^{k+i+1}(A,K_{i+1})$ in the long exact sequences for all the short exact pieces making up $E_*$.
 
 In particular, we can look at the image of $\id_M\in \ext^0(M,M)$ in $\ext^n(M,N)$. This is the map from extensions of length $n$ to $\ext^n$. Call this map $\phi$. Note that $\phi$ is well defined by the naturality of the connecting homomorphisms in the long exact sequences.
 
 Now we'll write down the inverse of $\phi$. Start with a class $[\alpha]\in \ext^n_R(M,N)$, so $\alpha\in \hom(P_n,N)$, where $(\cdots \to P_1\to P_0 \to M)$ is a projective resolution, and $\alpha$ induces the zero map in $\hom(P_{n+1},N)$ (under composition with $P_{n+1}\to P_n$). It follows that $\alpha$ factors through $K_{n-1}$, the cokernel of $P_{n+1}\to P_n$ (or the kernel of $P_n\to P_{n-1}$). But then we have an extension of length $n$
 \[
   0\to K_{n-1}\to P_{n-1}\to \cdots \to P_0\to M\to 0
 \]
 Pushing out along $\bbar\alpha\colon K_{n-1}\to N$, you get an extension of $M$ by $N$. You have to check that this extension is independent of cohomology class. To see that this procedure is onto, use something like the projective to acyclic lemma. You can go home and check that these two maps are natural and inverse.
\end{proof}

[[break]]

\begin{remark}
 Splicing extensions gives the \emph{Yoneda product}
 \[
  \ext^n_\A(M,N)\times \ext^m_\A(N,K)\to \ext^{m+n}_\A(M,K).
 \]
 Where $\A$ is an abelian category. This turns into the composition in the derived category via the isomorphism $\ext^n_\A(M,N)\cong \hom_{\D(\A)}^n(M,N)$.
\end{remark}

\subsektion{(square zero) Extensions of rings}

Let $\alpha\colon P\to R$ be a surjective ring homomorphism such that the kernel $J\subseteq P$ is a square zero ideal.
\begin{lemma}
 $J$ is an $(R,R)$-bimodule.
\end{lemma}
\begin{proof}
 $J$ is a $(P,P)$-bimodule because it is an ideal in $P$, and $J$ acts on itself by zero since $J^2=0$, so it is a $(P/J,P/J)$-bimodule.
\end{proof}
\begin{definition}
 Given a ring $R$ and an $(R,R)$-bimodule $J$, we define $\ext_\ring(R;J)$ to be the square zero extensions of $R$ by $J$ up to equivalence.
\end{definition}
\begin{lemma}
 Given an $(R,R)$-bimodule $I$ and a derivation\footnote{$d(rs)=d(r)s+rd(s)$.} $d\colon R\to I$, there is an induced map $\tilde d\colon \ext^1_{(R,R)\mod}(I,J)\to \ext_\ring(R;J)$.
\end{lemma}
\begin{remark}
 $(R,R)$-bimodules are the same thing as $R\otimes R^{op}$-modules, we $\ext$ makes sense. \anton{We can do $\ext$ in any abelian category.}
\end{remark}
\begin{proof}
 Take the pullback of an extension $0\to J\xrightarrow i E\xrightarrow \pi I\to 0$ along the derivation $d\colon R\to I$ to get some short exact sequence of abelian groups
 \[
  0\to J\to P\to R\to 0.
 \]
 The claim is that this is a square zero extension. To see this, we have to define the ring structure on $P$. Remember that $P=\{(e,r)|\pi(e)=d(r)\}$. The product structure is given by $(e,r)\cdot (e',r')=(e\cdot r'+r\cdot e',r\cdot r')$. I invite you to check that this still in $P$ (this uses the derivation property). The unit is $(0,1_R)$.
 
 $J$ is the set of elements of the form $(i(j),0)$, so it is clear that $J^2=0$ in $P$.
\end{proof}
\begin{remark}
 We had to go to bimodules because that is where derivations are naturally defined. Observe that if $R$ is a $k$-algebra (where $k$ is a commutative ring) we have an augmentation map of $k$-algebras $\e\colon R\to k$, then every left $R$-module can be turned into a bimodule via $m\cdot r = \e(r)\cdot m$.
\end{remark}
\begin{theorem}
 Let $\e\colon R\to k$ be an augmented $k$-algebra, with $I_r=\ker \e$ (thought of as an $(R,R)$-bimodule \emph{via $\e$}; well, for this theorem, we don't need the bimodule structure). For any $R$-module $J$ (turned into a bimodule via $\e$), there is an isomorphism $\ext^1_R(I_R,J)\cong \ext_\ring (R;J)$.
\end{theorem}
I'll prove this next time. I'll show you the map now. The idea is to pull back along some interesting derivation $R\to I_R$, given by $d(r)=r-\e(r)$. We'll check that this is a derivation.
\begin{theorem}
 $\ext_\ring(\ZZ G; J)$ is in bijection with group extensions of $G$ by $J$. This is $\ext^1_{\ZZ G}(I_G;J)=\ext^2_{\ZZ G}(\ZZ, J)\cong H^2(G;J)$.
\end{theorem}


