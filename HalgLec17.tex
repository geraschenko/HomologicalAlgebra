\sektion{17}{Group extensions}
\footnote{This lecture was given by Chris Schommer-Pries.}

Today we get to talk about one of the highlights of the course; we'll prove and understand something we couldn't understand before. We have a classification of finite simple groups, but then what do we know about finite groups? We need to understand how group extentions work.

Given groups $Q$ and $N$, when can we have an exact sequence of groups
\[
 1\to N\to E\to Q\to 1
\]
We can make a few observations right away. If $N$ is abelian, we kind of understand this problem. It turns out to be $H^2(Q;N)$. In general, we know that $N\triangleleft E$ and $Q=E/N$. So conjugation by elements of $E$ gives automorphisms of $N$, so we have a map $E\to \Aut N$. Before, if we had an element $q\in Q$, we took a preimage $e\in E$ and conjugate with it. But now that $N$ is non-abelian, the action of $q$ depends on the choice of preimage $e$ (a different choice will act by something differing by an inner automorphism of $N$). So we get
\[\xymatrix @R-1pc{
 1\ar[r] & N\ar[r] \ar@{=}[d] & E\ar[d]\ar[r] & Q\ar[r] \ar[d]^\rho & 1\\
 & N\ar[r]^-\alpha & \Aut(N)\ar[r] & Out(N) \ar[d]\\
 & & & \Aut(Z(N))
}\]
The goal for the day is to prove the following theorem.
\begin{theorem}
 Given $N$, $Q$, and $\rho\colon Q\to Out(N)$,
 \begin{itemize}
  \item[(a)] there is a functorial \emph{obstruction class} $\nu_\rho\in H^3(Q;Z(N))$ such that there exists an extension $E$ inducing $\rho$ if and only if $\nu_\rho=0$, and
  \item[(b)] if $\nu_\rho=0$, and $E_0$ is an extension inducing $\rho$, then $\{$extensions$\}/\cong$ is in bijection with $H^2(Q;Z(N))$.
 \end{itemize}
\end{theorem}
Consider the case of extensions by abelian groups $N=A$. In this case, $\alpha\colon A\to \Aut(A)$ is the trivial map. PT had an elaborite way of relating extensions to some other stuff, but there is a more direct way to do it, and that is what we'll do today.

If we have an extension
\[\xymatrix{
 1\ar[r] & A\ar[r] & E\ar[r] & Q\ar[r] \ar@/_/[l]_s & 1
}\]
we can choose a set-theoretic section $s$, which will not be a group homomorphism in general. We will have
\[
 s(q)s(q') = f(q,q')s(qq')
\]
for some function $f\colon Q\times Q\to A$. Computing $s(q)s(q')s(q'')$ in two different ways, we see that $f\in Z^2(Q;A)$. That is, \anton{explicitly write the cocycle condition}. If we'd chosen a different section, the two sections differ by a map $Q\to A$. Such a map is a $1$-cochain. The boundary of this $1$-cochain is exactly the difference between the resulting $2$-cocycles. So an extension gives us an element of cohomology. We can show that an element of cohomology gives an extension.

What changes in the non-abelian case? Suppose we have
\[\xymatrix@R-1pc{
 & 1\ar[r] & N\ar[r]\ar@{=}[d] & E\ar[r]_\pi\ar[d] & Q\ar[r]\ar[d]^\rho \ar@/_/[l]_s \ar[dl]|\xi & 1\\
 0\ar[r]& Z(N)\ar[r]& N\ar[r] & \Aut(N)\ar[r] & Out(N)
}\]
Composing the section with $E\to \Aut(N)$, we get a map $\xi$ (which won't be a homomorphism). In the abelian case, $\Aut(N)\cong Out(N)$, so we had to have $\xi=\rho$. As before, we get a function $f\colon Q\times Q\to N$, so that
\[
 \xi(q)\xi(q')=\alpha\bigl(f(q,q')\bigr)\xi(qq').
\]
Then we can verify that $f$ is a \emph{non-abelian cocycle}. That is,
\[
 f(q,q')f(qq',q'') = {}^{\xi(q)}[f(q',q'')]f(q,q'q'').
\]
If you happen to find an $f$ and a $\xi$ that satisfy this condition, then you can build an extension. You just take $E=N\times Q$ (as a set), and define the multiplication using your $f$ and $\xi$. However, it is not clear that you can find a $\xi$ and an $f$ satisfying these conditions (the cocycle condition and $\xi$ extends $\rho$ and $s$).
\begin{remark}[You can use your favorite $\xi$]
 If you had two different lifts $\xi$ and $\xi'$ of $\rho$, they will differ by some inner automorphism, which is induced by some element of $N$. Then we can change $s$ by that element of $n$, then we've changed $\xi'$ into $\xi$. So we can fix $\xi$ and ask, ``does such an $f$ exist?''
\end{remark}
Suppose for the time being that we've found such an $f$ somehow. How unique is this $f$? That is, how do we tell when two different $f$'s give the same extension. Before, we could change our section around and see how $f$ changes. But now we've already chosen our favorite $\xi$, so we can only try to change $s$ without changing $\xi$. Changing $s$ is the same as giving a map from $Q$ to $N$ (then $s$ changes by the image of this map under $\alpha$). Thus, we can change $s$ by any element in the kernel of $\alpha$ without messing with $\xi$. That is, we can change $s$ by any map from $Q$ to $Z(N)$ (i.e. a $1$-cochain). So this gives us one way to get two equivalent $f$'s. 

What if somebody gives us $(f,\xi)$ and $(\tilde f,\xi)$, and we want to know if they give the same extension. How do we tell? Consider the difference $f(q,q')^{-1}\tilde f(q,q')$. Because of the conditions $f$ and $\tilde f$ must satisfy, we must have that $f(q,q')^{-1}\tilde f(q,q')$ is in the kernel of $\alpha$, which is $Z(N)$. The upshot is that any two $f$'s are related by a function $\beta\colon Q\times Q\to Z(N)$. Furthermore, because of associativity in $Z(N)$, $\beta$ must satisfy a cocyle condition, so $\beta\in Z^2(Q;Z(N))$. By the previous paragraph, we know that $2$-coboundaries give the same extension. Since we know how to go from cocycles to extensions in a way that inverts this, we've basically proven part (b) of the theorem.

[[break]]

As you can see, these cocycles can very quickly connect extensions with $H^2$, but they are ``dirty.'' We'll use the cocycle stuff a little more, but then we'll have a more elegant way to think about it.

\begin{definition}
 A \emph{crossed module} is a pair of group homomorphisms $\partial\colon K\to E$ and $\rho\colon \to \Aut(K)$ such that
 \begin{itemize}
  \item[(1)] The diagram $\xymatrix @R-2pc @C-1pc{
   & E\ar[dr]^-\rho \\
   K\ar[ur]^\partial \ar[rr]_\alpha & & \Aut(K)
  }$ commutes, and
  \item[(2)] $\xymatrix@R-1pc{E \ar[r]^\rho \ar[d]_\alpha & \Aut(K) \ar[d]^{\partial_*}\\ \Aut(E) \ar[r]^{\partial^*} & \hom(K,E)}$ commutes (i.e.~$\partial({}^{\rho(e)}k)=e\partial(k) e^{-1}$)
 \end{itemize}
\end{definition}
Some facts. (1) If $A=\ker(\partial)$, then $A$ is abelian and contained in the center of $K$ (by condition (1)). (2) the image of $\partial$ is normal \anton{by condition (2)?}; call the quotient $Q$. (3) $E$ preserves $A$ via the action $\rho$ (by condition (2)). Moreover, we see that the action of $E$ on $A$ factors as $E\to Q\to \Aut(A)$.
\begin{example}
 If $N\triangleleft E$, then $E$ acts on $N$. In this case, $A$ is trivial.
\end{example}
\begin{example}
 We have $N\xrightarrow{\alpha} \Aut(N)$, which acts on $N$. In this case, $Q=Out(N)$ and $A=Z(N)$.
\end{example}
\begin{example}
 If $K\twoheadrightarrow E$ has abelian kernel, it is another example.
\end{example}
\begin{example}
 If $F\to E\to B$ is a fibration, you get a long exact sequence in homotopy groups, and the map $\pi_1(F)\to \pi_1(E)$ is an example of a crossed module.
\end{example}
\begin{example}
 If $Y\subseteq X$ is a CW pair, there is a long exact sequence in relative homotopy groups, and $\pi_2(X,Y)\to \pi_1(Y)$ is a crossed module.
\end{example}
Given a crossed module $K\xrightarrow\partial E$, we have
\[\xymatrix@R-1pc{
 0\ar[r] & A\ar[r]\ar[d] & K\ar@{=}[d]\ar[r]^\partial & E\ar[r]_\pi\ar[d] & Q \ar[d]\ar[r] \ar@/_/[l]_s & 1\\
 0\ar[r] & Z(K)\ar[r] & K\ar[r] & \Aut(K)\ar[r] & Out(K) \ar[r] & 1
}\]
You can choose a section $s$ as before. The resulting $f\colon Q\times Q\to \ker \pi$ gives us an extension
\[
 1\to \ker \pi \to E\xrightarrow \pi Q\to 1
\]
but that's not what we want ($\ker \pi$ is properly contained in $K$ if $A\neq 0$). We can try to find a lift $F\colon Q\times Q\to K$ so that $\partial F=f$. In general, $F$ \emph{will not} satisfy the cocycle condition, but it fails in a controlled way. We get
\[
 c(q,q',q'') F(q,q')F(qq',q'')={}^{\xi(q)}[F(q',q'')]F(q,q'q'')
\]
for some $c\colon Q\times Q\times Q\to A$. We then can check that $c$ satisfies a cocycle condition (because the action of $E$ on $A$ factors through $Q$?), so $c\in Z^3(Q;A)$. If you chase through it, you see that $c$ is only defined up to a coboundary. So given a crossed module, you can get an element of $H^3(Q;A)$.

Suppose we had two crossed modules with the same $A$ and $Q$, and suppose you have a map between them
\[\xymatrix@R-1.5pc{
 & & K\ar[r] \ar[dd]^\phi& E \ar[dr] \ar[dd]_\phi\\
 0\ar[r] & A\ar[dr]\ar[ur] & & & Q\ar[r] \ar@/_/[ul] & 1\\
 & & K'\ar[r] & E'\ar[ur]
}\]
Then you chase through it and you see that you get the same cocycle. This generates an equivalence relation on crossed modules. Let $CM(Q;A)$ be such crossed modules up to this equivalence. So we've shown that if we have a map $CM(Q;A)\to H^3(Q;A)$.

The following theorem will imply part (a) if the theorem.
\begin{theorem}
 $CM(Q;A)\to H^3(Q;A)$ is a bijection. In fact, it is an isomorphism of groups.
\end{theorem}
The idea is to first show that crossed modules are functorial. If you have a crossed module, you can pull them back along a map $Q'\to Q$:
\[\xymatrix@R-1pc{
 0\ar[r] & A\ar[r] \ar@{=}& K\ar[r]\ar@{=}[d] & E' \ar@{}[dr]|(.25)\pb \ar[r] \ar[d] & Q' \ar[r]\ar[d] & 1\\
 0\ar[r] & A\ar[r] & K\ar[r] & E\ar[r] & Q \ar[r] & 1
}\]
Similarly, you can push forward along maps $A\to A'$. Now you can define an addition on $CM(Q;A)$. Given two crossed modules, you take their direct sum, pull back along the diagonal map $Q\to Q\times Q$ and push forward along the sum map $A\to A\oplus A$ (this is a group homomorphism because $A$ is abelian). This makes $CM(Q;A)$ into an abelian group.

With a little more care, you can see that the map $CM(Q;A)\to H^3(Q;A)$ is compatible with the addition in $H^3$, so this is a morphism of groups.

We want to show that the map is an isomorphism. The interesting part is to show that the kernel is trivial. We want to show that the zero in $H^3$ is only hit by ``the trivial crossed module''. The trivial crossed module is
\[
 0\to A=A\xrightarrow{const} Q=Q\to 1
\]
Suppose we have a crossed module that goes to zero in $H^3$.
\[\xymatrix{
 0\ar[r] & A \ar@{=}[r] & A\ar[r] & Q\ar@{=}[r] & Q\ar[r] & 1\\
 0\ar[r] & A\times 1\ar@{=}[u]\ar[r]\ar@{=}[d] & A\times K\ar[u]\ar[d]\ar[r] & \tilde E\ar[d]\ar[u]\ar[r] & Q\ar@{=}[d]\ar@{=}[u] \ar[r] & 1\\
 0\ar[r] & A\ar[r] & K\ar[r]^\partial & E\ar[r] & Q\ar[r] & 1
}\]
The $c\in Z^3$ we get from $F\colon Q\times Q\to K$ is trivial (this is the assumption), so we get the middle row (we've thrown in a little extra $A$ in the first two terms)

We've actually done
\[\xymatrix @R-.5pc{
 0\ar[r]& 1\ar[r]\ar[d]& N\ar@{=}[d] \ar[r] & E\ar[r] \ar@{.>}[d] & Q\ar@/^3ex/[dd]^\rho\\
 0\ar[r]&Z(N)\ar[r]\ar@{=}[d] & N\ar@{=}[d]\ar[r] & \tilde E\ar@{}[dr]|(.25)\pb \ar[r]\ar[d] & Q\ar[d]_\rho \ar@{=}[u] \\
 0\ar[r]&Z(N)\ar[r] & N\ar[r]^-\alpha & \Aut(N)\ar[r] & Out(N)
}\]
If we had an extension $E$, then we get a map of crossed modules to the canonical pullback $\tilde E$.

Inside $H^3(Out(N);Z(N))$, there is a canonical class $\alpha$ corresponding to the crossed module $\Aut(N)$ (bottom row). We've shown that there exists an extension only if the pullback class in $H^3(Q;Z(N))$ is zero. On the other hand, \anton{the other way somehow}





