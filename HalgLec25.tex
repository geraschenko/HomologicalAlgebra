\sektion{25}{Universal coefficients} 

Today: some more spectral sequences and applications. Thursday: Derived categories and derived functors.

So far, I've only shown you the Leray-Serre spectral sequence, but in the homework, you've seen the spectral sequence associated to a double complex. Recall that if $(D_{p,q},d^v,d^h)_{p,q\ge 0}$ is a double complex, then you get two spectral sequences converging to the homology of the total complex $\tot_{p+q}(D_{*,*})$ (but with different filtrations!):
\begin{align*}
 E^0_{p,q}&=(D_{p,q},d^v)\tag{I}\\ 
 E^0_{p,q}&=(D_{q,p},d^h)\tag{II}
\end{align*}
The second spectral sequence is not a cohomological spectral sequence, it is a homological spectral sequence transposed.
\begin{example}
 Say $P_*\twoheadrightarrow M$ is a projective resolution of some (right) $R$-module $M$, and $Q_*$ is some other chain complex of projective (left) $R$-modules. Let $D_{*,*}=(P_*\otimes_R Q_*, d_P,d_Q)$. The first spectral sequence gives us (using that $Q_*$ has projective terms to get the $E^2$ term)
 \[
  E^1_{p,q}= P_p\otimes H_q Q_*\qquad E^2_{p,q}= \tor_p^R(M,H_q Q_*) 
 \tag{I}\]
 \[
  E^1_{p,q}= H_q(P_*) \otimes Q_p =
  \begin{cases}
   M\otimes Q_p & p=0\\
   0 & p>0
  \end{cases} \qquad E^2_{p,q} = E^\infty_{p,q}= H_p(M\otimes Q_*)
 \tag{II}\]
 So we know that $\tor^R_p(M,H_q Q_*)\Rightarrow H_{p+q}(\tot) = H_{p+q}(M\otimes Q_*)$.
\end{example}
In the case $R=\ZZ$, $Q=C_*(X)$ for a topological space $X$, and $M$ some abelian group, then we get (we can flip our tensor products) the \emph{homological universal coefficients spectral sequence}
\[
 E^2_{p,q}=\tor_p^\ZZ(H_q X,M)\Rightarrow H_{p+q}(C_* X\otimes M).
\]
The spectral sequence is concentrated at $p=0,1$ because $\ZZ$ is a PID, so we are in the first two columns. \anton{But the differentials change $p$ by 2, so we have that $E^2=E^\infty$. The conclusion is that there is a short exact sequence
\[
 0\to \tor_1^\ZZ(H_{q-1}X,M)\to H_q(X;M)\to H_q X\otimes_\ZZ M\to 0
\]}
The edge homomorphism at $p=0$ is $E^2_{0,q}=M\otimes_R H_qQ_*\to H_q(M\otimes_R Q_*)$

When will I use the more general spectral sequence? Suppose $G=\pi_1 X$ and $M$ is a $G$-modules. Then you may want to compute $H_*(X;M):=H_*(C_*\tilde X\otimes_{\ZZ G}M)$. Now we have the spectral sequence $\tor_p^{\ZZ G}(H_q \tilde X,M)\Rightarrow H_{p+q}(C_*\tilde X\otimes_{\ZZ G}M)$. You might call this the universal coefficients spectral sequence for twisted coefficients.

You know another universal coefficient theorem, for cohomology. It can be expressed as a spectral sequence
\[
 \ext^p_\ZZ(H^qX,M)\Rightarrow H^{p+q}(X;M)
\]
Again, the spectral sequence is concentrated in the columns $p=0,1$. The more general \emph{cohomological universal coefficients spectral sequence} is
\[
 E_2^{p,q}=\ext^p_R(H_q Q_*,M)\Rightarrow H^{p+q}(Q_*;M):=H_{p+q}(\hom_R(Q_*,M))
\]
The proof is done by considering $D_{p,q}=\hom_R(Q_q,P_p)$, where $M\to P_*$ is an injective resolution and $Q_*$ is a complex of projectives. The edge homomorphism at $p=0$ is
\[
 H^q(Q_*,M)=H_q(\hom_R(Q_*,M))\to E^2_{0,q}=\hom_R(H_q Q_*,M)
\]
If $Q_*=C^*X$ for some topological space $X$ and $M$ is an interesting $\pi_1X$-module, then you really need the big spectral sequence.

In HW12, you'll do the K\"unneth spectral sequence, which generalizes the K\"unneth theorem in the same way that the universal coefficients spectral sequences generalize the universal coefficients theorems.

After the break, we'll calculate $\pi_{n+1}S^n$. In the next homework, you'll compute all homotopy groups of spheres rationally ($\pi_k S^n\otimes_\ZZ \QQ$). This stuff is due to Serre.

[[break]]

\subsektion{\texorpdfstring{$\pi_{n+1}(S^n)$}{pi{n+1}(S^n)}}

\begin{theorem}[Serre]
 $\pi_{n+1}(S^n)\cong H_{n+2}(K(\ZZ,n))\cong
 \begin{cases}
  0 & n=0,1\\
  \ZZ & n=2\\
  \ZZ/2 & n\ge 3
 \end{cases}$
\end{theorem}
It's true in general that $\pi_{n+k}(S^n)$ stabilizes (more generally, $\pi_{n+k}(\Sigma^n X)$ stabilizes) for large $n$ (Fruedenthal's suspension theorem, in fact, it is stable by $n=2\dim X -1$). This can also be proven easily by induction with spectral sequences using the path-loop fibration.
\begin{proof}
 Assume $n>1$ (we know the result for $n=0,1$). We know that $\pi_i(S^n)=0$ for $i<n$ and that $\pi_n(S^n)\cong \ZZ$. By adding cells of dimenions at least $n+2$ we construct an inclusion map $S^n\to K(\ZZ,n)$ inducing an isomorphism on $\pi_n$. We can turn this into a fibration (changing $S^n$ to something homotopy equivalent; we won't change the notation). Let $F$ be the fiber. Using the long exact sequence of homotopy groups, we get $\pi_{n+1}(S^n)\cong \pi_{n+1}(F)$. Note that $F$ is $n$-connected because of the long exact sequence in homotopy groups. So by the Hurewicz theorem, $\pi_{n+1}(F)\cong H_{n+1}(F)$. The Leray-Serre spectral sequence tells us that $H_{n+1}(F)\cong H_{n+2}(K(\ZZ,n))$.
 \[\xymatrix@!0 @R+0pc @C+0pc{
 n+1\qquad & H_{n+1}F\\
 n  & 0\\
 & \vdots \\
 & 0 & & \smash{\mbox{\Huge 0}}\\ 
 0 & \ZZ & 0 & \cdots & 0 & \ZZ & 0 & \rlap{$H_{n+2}(K(\ZZ,n))$}\ar[lllllluuuu]\\
 & 0 & & & & n & & n+2
 }\]
 Note that $H_{n+1}(K(\ZZ,n))=0$ because there are no $(n+1)$-cells. All the differentials are zero except for the one shown, which must be an isomorphism because we know the homology of $S^n$.

 Now we have to compute $H_{n+2}(K(\ZZ,n))$. The special case is $n=2$. This ``easy way'' is to observe that $\CC P^\infty$ has the right homtopy groups (you have to know this). The other way is to take the path space fibration $\Om_k K\to P_k K\to K$ for some base point $k\in K$. Applying the Leray-Serre spectral sequence to the case $K=K(\ZZ,n)$, where $\Om_k K=K(\ZZ,n-1)$, we can do an induction.
 
 The first fibration is $S^1\cong K(\ZZ,1)\to P\simeq 1\to K(\ZZ,2)$. The spectral sequence is
 \[\xymatrix @!0{
  \vdots\\
  2& 0 & 0 & 0 & \cdots\\
  1& \ZZ & 0 & \ZZ & 0 & H_4K & \cdots\\
  0& \ZZ & 0 & \ZZ\ar[ull] & 0 & H_4K\ar[ull]\\
  & 0 & 1 & 2 
 }\]
 We know that the two rows have to be equal (they are both $H_n(K;\ZZ)$), and the $E^\infty$ page has to be full of zeros, so all the $E^2$ differentials have to be isomorphisms, so $H_{2n}(K(\ZZ,2))\cong \ZZ$.

 Now consider the fibration $K(\ZZ,2)\to *\to K(\ZZ,3)$. Then we get the spectral sequence
 \[\xymatrix@!0 @-.3pc{
  4&\ZZ & 0 & 0 & \ZZ & 0 & H_5K\\
  3&0 & 0& 0& 0& 0& 0\\
  2&\ZZ & 0 & 0 & \ZZ \ar[uulll]|{d_3} & 0 & H_5K\\
  1&0 & 0& 0& 0& 0& 0\\
  0&\ZZ & 0 & 0 & \ZZ \ar[uulll] & 0 & H_5K\\
   & 0 & 1 & 2 & 3 & 4 & 5
 }\qquad
 \xymatrix@!0 @-.3pc{
  4&? & 0 & 0 & \ZZ & 0 & H_5K\\
  3&0 & 0& 0& 0& 0& 0\\
  2&0 & 0 & 0 & E^4 & 0 & H_5K\\
  1&0 & 0& 0& 0& 0& 0\\
  0&\ZZ & 0 & 0 & 0  & 0 & H_5K \ar[uuuulllll]\\
   & 0 & 1 & 2 & 3 & 4 & 5
 }\]
 The non-zero rows have to be equal ($\pi_1(K(\ZZ,3))=0$, so there cannot be a non-trivial action, so they are all just homology with coefficients in $\ZZ$). The goal is to compute that $d_3$ is multiplication by 2. The key fact will be the Pontrjagin product on $H_*(K(\ZZ,n))$. Serre did it with cohomology an used cup products, but this is a good approach to keep in mind.
\end{proof}




